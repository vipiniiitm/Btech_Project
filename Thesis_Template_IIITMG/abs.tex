\newpage
\newpage
\addcontentsline{toc}{chapter}{ABSTRACT}
\begin{center}
{\large \bf ABSTRACT}
\end{center}
In recent years social media like Twitter has become one of the major source for opinion and information sharing and this became popular with the affordable mobile devices and its portability with social media applications . People on social media like Twitter share their view on many topics and these informations are mined for various applications and predictions . One such application is real time disease surveillance . In this project we have done real time disease surveillance regarding zika virus on Twitter .  We have used Twitter API and various Python libraries to collect tweets on Zika virus from specific geographical location using a longitude , latitude value as centre and radius value in order to cover desired area . We have used various Python libraries like Natural Language Toolkit (NLTK) and Naive Bayes Algorithm in order to find polarity of tweets and hence finding out level of concern among people from particular location (USA).We have analysed the tweets based on six phrases in order to decide awareness level among people regarding Zika virus. Finally we have found out based on percentage of negative tweets that now people are less concerned about Zika virus as expected.

{\it Keywords:} - Twitter , Sentiment Analysis , Zika Virus , Disease surveillance.